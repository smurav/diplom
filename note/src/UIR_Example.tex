\documentclass{../class/UIR}

% БИБЛИОГРАФИЯ
\bibliographystyle{utf8gost71s}

% Свойства для титульной страницы
\Title          {Реализация инфраструктуры открытых ключей \break для аутентификации клиентов СУБД \break в распределенной информационной системе}
\StudentGroup   {К10-361}
\StudentName    {Лаврентьева М.О.}
\SupervisorPost {Доцент ,к.т.н.}
\SupervisorName {Муравьёв С.К.}
\CathedraName {Бельтов А.Г.}

\begin{document}
 
% Титульная страница
\maketitle

% Задание на УИР
\maketz 

% Содержание
\tableofcontents
\StructureElement{Введение}
В настоящее  время при осуществлении практически любой деятельности человек сталкивается с необходимостью искать, хранить, обрабатывать и передавать информацию.
Сегодня информация  -  один из основных ресурсов развития общества, а информационные системы (ИС) и технологии - средство повышения производительности и эффективности работы. 
Информационная система – это взаимосвязанная совокупность средств, методов и персонала, используемых для хранения, обработки и выдачи информации в интересах достижения поставленной цели. Основной задачей ИС является удовлетворение информационных потребностей в рамках конкретной предметной области. 
В научной, учебной, производственной, управленческой и финансовой деятельности широко используются самые разнообразные информационные системы, ведется их разработка, совершенствование, внедрение и активное использование в повседневной деятельности.
В идеале в рамках предприятия функционирует единая корпоративная информационная система, удовлетворяющая  потребностям, как отдельных сотрудников, так и потребностям служб и подразделений.
Такие большие информационные системы требуют существенной  мощности локальной вычислительной сети, состоящей из множества рабочих станций выполняющих разные задачи и имеющих разные функции. 
Информационные системы, которые позволяют распределить процессы для их хранения, обработки и представления по различным компьютерам называются распределенными. 
В состав небольших информационных систем может входить всего несколько рабочих станций, за которыми работают определенные пользователи. При изменении конфигурации одного из пользователей или при появлении нового пользователя администратору информационной системы легко внести изменения на всех рабочих станциях непосредственно вручную.
 Современные компании и предприятия используют информационные системы, состоящие из десятков, сотен рабочих станций, на которых может работать большое количество пользователей с различными правами доступа к ресурсам системы. Для таких информационных систем важным пунктом их организации является вопрос централизованного администрирования, при котором управление доступом ведется с центрального компьютера.
При этом не стоит забывать, что любая распределенная информационная система подвержена угрозе информационной безопасности, а именно несанкционированным воздействиям и доступу посторонних лиц или программ к обрабатываемой информации. Результатом такого воздействия может стать потеря целостности и достоверности хранимой информации. Поэтому обязательной составной частью современной распределенной информационной системы является система обеспечения информационной безопасности.

Задачи данной дипломной работы:
\begin{enumerate}
\item Разработать  архитектуру распределённой ИС с возможностью централизованного администрирования.
\item Реализовать безопасный доступ к разделяемому удалённому ресурсу – серверу СУБД PostgreSQL.
\item Предложить  и реализовать решение проблемы обеспечения информационной безопасности распределённой ИС.
\end{enumerate}



\section{Описание распределённой ИС}	

\subsection{Постановка задачи}
\subsection{Анализ технологий}
\subsection{Описание технологий}

\section{ГЛАВА}
\subsection{часть}
\subsection{часть}
\subsection{часть}

\section{ГЛАВА}
\subsection{часть}
\subsection{часть}
\subsection{часть}
		
% Заключение   
\StructureElement{Заключение}


\StructureElement{Список используемой литературы}
\begin{enumerate}
\item 
\end{enumerate}


% Отзыв руководителя
\begin{ReviewOfSupervisor}

\end{ReviewOfSupervisor}

\end{document}


